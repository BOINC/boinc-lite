\documentclass[a4paper,11pt]{article}
\usepackage[pdftex]{graphicx} 
\author{Peter Hanappe \\ Sony Computer Science Laboratory \\ hanappe@csl.sony.fr}
\title{BoincLite: A thin BOINC client library}

%*****************************************************************
%*****************************************************************
\begin{document}
\maketitle
  
BoincLite defines the minimum API to realise a simple BOINC
client. Its main functionality is to download work units, to upload
the results, and to make sure there is always one work unit ready to
be executed. It has a fair number of restrictions compared to the
existing BOINC manager:

\begin{itemize}

\item It handles only one BOINC project.

\item It doesn't manage the resource usage (CPU, memory, disk) nor the
  scheduling between projects and work units. If resource management
  is needed, the calling application will have to implement it.

\item It doesn't provide any glue for constructing a user
  interface. Again, this has to be managed by the calling application.

\item Only one workunit is computed at once and at most one workunit
  is waiting to be executed.

\item The downloads and uploads are performed sequentialy.

\item The library doesn't use multi-threading. However, it is
  straighforward to spin of separate threads for the scheduler for and
  the computation of the work unit so that they run parallel to the
  main thread.

\end{itemize}

\begin{figure}[h!]
  \label{fig:components}
  \vspace{10pt}
  \centering{\includegraphics[width=0.6\linewidth]{components}}
  %\caption{}
\end{figure}

The library consists of four components: BoincScheduler, BoincProxy,
BoincConfiguration, and BoincHttp (see Fig.~\ref{fig:components}).

BoincHttp is a thin abstraction layer for handling HTTP requests. It
is used as a portable layer that can be easily implemented on top of
existing HTTP libraries, such as libcurl on Linux or libhttp on
CellOS.

The BoincConfiguration is another shallow abstraction layer. It
manages all the configuration data needed by BOINC, such as the user
ID, the authenticator string, but also the disk usage or the FLOPS of
the host machine.

BoincProxy handles the RPC messages with the BOINC server. It uses
expat to parse the XML data. It relies on BoincHttp and
BoincConfiguration for all the platform specific functions.

The principal functionality of the BoincScheduler is assure that there
is always one work unit available to be computed. It is possible to
overlap the computation with the network transfers but it requires
threading support from the calling application. The scheduler also
handles the loading/storing of work units from/to disk. At any stage
during the life span of a work unit, the scheduler should store the
information of a work unit on disk. The BoincScheduler calls upon the
BoincProxy to request new work units or to upload the results and uses
BoincHttp to download the data of the work unit.

\begin{figure}[h!]
  \label{fig:workunit-scheduling}
  \vspace{10pt}
  \centering{\includegraphics[angle=90,width=0.25\linewidth]{workunit-scheduling}}
  %\caption{}
\end{figure}


\begin{figure}[h!]
  \label{fig:workunit-states}
  \vspace{10pt}
  \centering{\includegraphics[width=0.6\linewidth]{workunit-states}}
  %\caption{}
\end{figure}



\end{document}
